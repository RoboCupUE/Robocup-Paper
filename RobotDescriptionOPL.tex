\section*{Annex}
\subsection*{TurtleBot 4 Specifications[OPL]}%
\label{sec:annex-SSPL}
% In this section briefly describe the software and hardware of the robot

\setlength\intextsep{0pt}
\begin{wrapfigure}[10]{r}{0.3\textwidth}
	\centering
	\includegraphics[width=0.4\textwidth]{images/robot.jpg}
	\caption{TurtleBot 4}%
	\label{fig:turtlebot4}
\end{wrapfigure}

The TurtleBot 4, developed by \textit{Clearpath Robotics}, is a modular and open platform robot designed for research and education. In this section, we detail the hardware and software configurations employed by Tech-bROS to enhance the TB4's functionality for dynamic service tasks and human-robot interaction.

\subsection*{Hardware Description}
\begin{itemize}
    \item \textbf{Dimensions and Weight:} 
        \begin{itemize}
            \item Diameter: 320 mm.
            \item Height: 430 mm.
            \item Weight: 10.5 kg.
        \end{itemize}
    \item \textbf{Base:} Differential drive system with two active wheels and a passive caster for stability.
    \item \textbf{Sensors:}
        \begin{itemize}
            \item LIDAR: 360° scanning with RPLIDAR A1.
            \item Depth Camera: OAKD-Pro for RGB-D perception and spatial mapping.
            \item IMU: Built-in Inertial Measurement Unit for motion tracking.
            \item Bumper and Cliff Sensors: Obstacle detection and edge safety.
        \end{itemize}
    \item \textbf{Actuators:} 
        \begin{itemize}
            \item Two DC motors with integrated encoders for precise movement.
        \end{itemize}
    \item \textbf{Power:} Lithium-Ion battery providing 2–3 hours of continuous operation.
    \item \textbf{Processor:} Raspberry Pi 4B with 4 GB RAM.
\end{itemize}

\subsection*{Software Description}
\textit{For our robot we are using the following software:}
\begin{itemize}
    \item \textbf{Operating System:} ROS 2 Humble Hawksbill on Ubuntu 22.04 LTS.
    \item \textbf{Localization and Navigation:}
        \begin{itemize}
            \item Adaptive Monte Carlo Localization (AMCL).
            \item Nav2 Stack for SLAM and navigation.
        \end{itemize}
    \item \textbf{Perception:}
        \begin{itemize}
            \item Object detection using YOLOv5, \textit{Darknet ROS} and \textit{Darknet ROS 3D}.
            \item SLAM with Cartographer for simultaneous localization and mapping.
        \end{itemize}
    \item \textbf{Human-Robot Interaction:}
        \begin{itemize}
            \item Object and facial expression recognition through depth camera integration.
        \end{itemize}
    \item \textbf{Simulation and Testing:} 
        \begin{itemize}
            \item Custom Gazebo worlds for benchmarking and validation.
            \item Real-world pre-mapped environments within our institution.
        \end{itemize}
    \item \textbf{Data Logging:}
        \begin{itemize}
            \item ROSbag for recording experiment data.
        \end{itemize}
\end{itemize}

\subsection*{System Customizations}
\noindent\textit{Also our robot incorporates the following devices:}
\begin{itemize}
    \item \textbf{Additional Hardware:}
        \begin{itemize}
            \item NVIDIA Jetson X2 for AI inference and deep learning tasks.
            \item External WiFi module for enhanced cloud connectivity.
            \item Robotic arm mounted on top shelf as a means of interaction with the surrounding environment.
        \end{itemize}
    \item \textbf{Software Enhancements:}
        \begin{itemize}
            \item Custom plugins for Nav2 to handle dynamic obstacles.
            \item Extended SLAM capabilities with multi-sensor fusion.
            \item Custom behavior trees to improve the robustness of the robot autonomy.
        \end{itemize}
\end{itemize}